\verbatim{
-artig       du mot Art, la manière
-bach        rivière et fleuve
-bar         sert à former un adjectif à partir d'un verbe,
             et indiquant la capacité à lare l'action décrite par celui-ci.
             -able (ou -ible) en français.
-chen        suffixe diminutif. Neutre
-ei          lieu d'exercice d'une profession
-er          habitant, et qui fait l'action, -eur ou -ier sont synonymes
-fach        le chiffre ou la quantité indiquée par le radical, fois
-frei        pour former des substantifs indiquant l'abscence d'une chose
-graphie     
-haltig      pour former des adjectifs à partir de substantifs, et indiquant la présence d'une chose
-heit        pour former des substantifs féminins à partir d'adjectifs,
             en indiquant la qualité décrite par ceux-ci.
-id          indique la masculinité
-ier-        suffixe inclus dans le radical pour les verbes importés
-ig          permet de transformer un nom en adjectif
-in          pour former des mots féminins
-innen       suffixe pour inclure les formes masculines et féminine
-innig       intime
-isch        permet de transformer un substantif en adjectif, signifiant: qui a la qualité de
             correspond au français -ique
-ismus       croyance, doctrine, qualité, particularité ou maladie
-ität        féminin
-keit        servant à former des substantifs féminins à partir d'adjectifs,
             et indiquant la qualité décrite par ceux ci.
-kratie      féminin
-kunde       équivalent à -logie
-lein        suffixe diminutif
-lich        sert à créer des adjectifs ou des adverbes à partir de substantifs ou de verbes,
             et signifiant en gros "de manière"
-los         suffixe servant à créer des adjectifs à partir de substantifs,
             et signifiant l'absence de quelque chose, une privation (arbeitlos)
-macher      noms de métier dans le cas d'un fabricant (masculin)
             celui qui fait, le faiseur de
-n           permet d'obtenir des verbes à partir de certains adjectifs
-nis         permet de transformer un verbe en un substantif neutre,
             qui devient la matérialisation de celui-ci.
-on          physique
-pathie      sentiment, autres
-s           génitif singulier, marque de pluriel, épenthèse utile pour la formation de mots composés
-saurier     zoologie
-’sch        suffixe apposé au nom d'une personne pour former un adjectif.
             l'adjectif dérivé s'écrit en majuscule
-sch         Suffixe apposé au nom d'une personne pour former un adjectif.
             l'adjectif dérivé s'écrit en minuscule
-schaft      sert à former des substantifs féminins à partir d'autres substantifs ou de verbes,
             et indiquant la qualité décrite par ceux-ci
-st          suffixe utilisé pour former un ordinal à partir d'un cardinal de BX ou plus
-ste         suffixe utilisé pour former un ordinal à partir d'un cardinal de BX ou plus.
             Le plus couramment utilisé.
-t           suffixe utlisé pour construire le participe passé des verbes faibles
             à préfixe non accentué.
             utilisé également pour former un ordinal à partir d'un cardinal moins de BX
-te          utilisé également pour former un ordinal à partir d'un cardinal moins de BX.
             le plus couramment utilisé.
-ung         transformation d'un verbe en substantif. Les substantifs sont alors féminins
-zid         équivalent du cide français. bakterizides -> bactéricide
}
